\documentclass{beamer}

\input{settings.tex}


\title{Linear inequality representation of convex domains}
\subtitle{Computational Intelligence, Lecture 7}
\author{by Sergei Savin}
\centering
\date{Spring 2021}

\begin{document}
\maketitle


\begin{frame}{Content}

\begin{itemize}
\item Convex polytopes
\item Half-spaces
\begin{itemize}
\item Definition
\item Construction. Simple case
\item Construction. General case
\item Combination
\item Formal description via inequalities
\end{itemize}
\item Linear approximation of convex regions
\item Homework
\end{itemize}

\end{frame}



\begin{frame}{Convex polytopes}
% \framesubtitle{Parameter estimation}
\begin{flushleft}

Before defining what a convex polytope is, let us look at examples:

\include{fig1}
 
\end{flushleft}
\end{frame}



\begin{frame}{Half-spaces}
\framesubtitle{Definition}
\begin{flushleft}

We can define half-space as a set of all points $\mathbf{x}$, such that $\mathbf{a}^\top \mathbf{x} \leq b$. It has a very clear geometric interpretation. In the following image, the filled space is \textbf{not} in the half space.

\include{fig2}
 
\end{flushleft}
\end{frame}



\begin{frame}{Half-spaces}
\framesubtitle{Construction. Simple case}
\begin{flushleft}

Consider half-space that passes through the origin, and defined by its normal vector $\mathbf{n}$:

\include{fig3}

It is easy to see that this half-space can be defined as "all vectors $\mathbf{x}$, such that $\mathbf{n} \cdot \mathbf{x} \leq 0$", which is the same as using $\mathbf{n}$ instead of $\mathbf{a}$ in our original definition, setting $b = 0$.
 
\end{flushleft}
\end{frame}




\begin{frame}{Half-spaces}
\framesubtitle{Construction. General case}
\begin{flushleft}

In the general case there is some distance between the boundary of the half-space and the origin, let's say $d$.

\include{fig4}

The same way we see, that the half space can be defined as "all vectors $\mathbf{x}$, such that $\mathbf{n} \cdot \mathbf{x} \leq d$". This is the same as making $\mathbf{a} = \mathbf{n}$ and $b = d$ in our original definition. However, if $\mathbf{a}$ is not a unit vector, $b = d ||\mathbf{a}||$.
 
\end{flushleft}
\end{frame}



\begin{frame}{Half-spaces}
\framesubtitle{Combination}
\begin{flushleft}

We can define a region of space as an \emph{intersection} of half-spaces $\mathbf{a}_i^\top \mathbf{x} \leq b_i$:

\include{fig5}

Resulting region will be easily described as $\begin{bmatrix} \mathbf{a}_1^\top \\ ... \\ \mathbf{a}_k^\top \end{bmatrix} \mathbf{x} \leq \begin{bmatrix} b_1 \\ ... \\ b_k \end{bmatrix}$

 
\end{flushleft}
\end{frame}


\begin{frame}{Half-spaces}
\framesubtitle{Formal description via inequalities}
\begin{flushleft}

The last result allows us to write any convex polytope as a matrix inequality:

\begin{equation}
\label{eq:ineq} 
    \mathbf{A} \mathbf{x} \leq  \mathbf{b} 
\end{equation}

And conversely, any matrix inequality \eqref{eq:ineq} represents either an empty set or a convex polytope.
 
\end{flushleft}
\end{frame}




\begin{frame}{Linear approximation of convex regions}
% \framesubtitle{Parameter estimation}
\begin{flushleft}
Some convex regions can be easily approximated using polytopes.

\include{fig6}

Which allows to represent constraints on $\mathbf{x}$ to belong in such a region as a matrix inequality
 
\end{flushleft}
\end{frame}



\begin{frame}{Homework}
% \framesubtitle{Parameter estimation}
\begin{flushleft}

Represent in matrix inequality form the following figures:

\begin{itemize}
    \item Equilateral triangle
    \item A square
    \item Parallelepiped
    \item Trapezoid
\end{itemize}

\end{flushleft}
\end{frame}


% \begin{frame}{Self-study}
% % \framesubtitle{Part 3}
% \begin{flushleft}

% \begin{itemize}
%     \item \href{https://www.youtube.com/watch?v=kcOodzDGV4c}{Convex Optimization, lecture 3, S. Boyd. Stanford. Convex functions}.
% \end{itemize}

% \end{flushleft}
% \end{frame}



\begin{frame}
\centerline{Lecture slides are available via Moodle.}
\bigskip
\centerline{You can help improve these slides at:}
\centerline{
\textcolor{blue}{\href{https://github.com/SergeiSa/Computational-Intelligence-Slides-Spring-2021}{github.com/SergeiSa/Computational-Intelligence-Slides-Spring-2021}}
}
\bigskip

\textcolor{black}{\qrcode[height=1.5in]{https://git.io/JYRBT}}
\bigskip

\centerline{Check Moodle for additional links, videos, textbook suggestions.}
\end{frame}

\end{document}
